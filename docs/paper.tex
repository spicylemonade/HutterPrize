\documentclass[11pt]{article}
\usepackage[a4paper,margin=1in]{geometry}
\usepackage{hyperref}
\usepackage{graphicx}
\usepackage{listings}
\usepackage{booktabs}
\usepackage{siunitx}
\usepackage{enumitem}
\title{Towards a New Hutter Prize Record on enwik9: System Design and Implementation}
\author{HutterPrize Solver Authors}
\date{\today}
\begin{document}
\maketitle
\begin{abstract}
Iteration 5 extends the reversible streaming transform layer with newline and digit run encoding in addition to dictionary tokenization and space-run encoding. We add CLI knobs to choose the compression backend (zlib or store) and to toggle transforms. The build remains dependency-free (dynamic zlib is loaded at runtime if available), and the self-extracting archive continues to be a single binary that reproduces enwik9 bit-identically. These changes deliver measurable S2 gains with zlib and strengthen the pipeline for future integration of a stronger backend and advanced modeling.
\end{abstract}

\section{Self-Extracting Format}
We concatenate a decompressor stub with a payload and an HPZ2 footer (magic, method byte, original size, payload size, CRC-32 of original). When transforms are enabled, the payload starts with an 8-byte HPZT header (magic ``HPZT'', version=1, flags) that declares active transforms.

\section{Transforms}
\begin{itemize}[noitemsep]
  \item Dictionary tokenization: frequent XML/Wikitext substrings are replaced by two-byte tokens (0x00, id).
  \item Space-run encoding: runs of spaces of length \(\ge 4\) are replaced by 0x00, 0x80, len$-$4.
  \item Newline-run encoding (new): runs of newlines of length \(\ge 2\) are replaced by 0x00, 0x81, len$-$2.
  \item Digit-run encoding (new): runs of digits of length \(\ge 3\) are replaced by 0x00, 0x82, len$-$3 followed by the digit bytes (saving one byte per run and improving compressibility).
  \item Escape: literal 0x00 is encoded as 0x00, 0x00.
\end{itemize}
All transforms are streaming and strictly reversible; decoding is driven by a simple finite-state machine.

\section{Compression Backend and CLI}
We use dynamic zlib (loaded at runtime) when available and fall back to STORE otherwise. The compressor now supports:\\
\texttt{--method=zlib|store} and \texttt{--no-transform}. This enables controlled experiments and ablations.

\section{Roadmap}
Next iterations will vendor a stronger bundled backend (e.g., LZMA or a static entropy coder) and introduce additional reversible transforms (numeric/date canonicalization and structural tagging), then integrate a context-mixing model. The final paper will include detailed ablations.

\end{document}
